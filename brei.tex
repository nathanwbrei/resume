% Resume of Nathan Brei
% Uses res.cls, courtesy of RPI

\documentclass[margin]{res}

\setlength{\topmargin}{-0.3in}  % Start text higher on the page 
\setlength{\textheight}{9.8in}  % increase textheight to fit more on a page
\setlength{\headsep}{0in}     % space between header and text
\setlength{\headheight}{12pt}   % make room for header
\setlength{\textwidth}{6in} % set width of text portion
%\setlength{\resumewidth}{5in}
\setlength{\hoffset}{0.75in}
\setlength{\voffset}{0in}
\setlength{\headheight}{0in}

\begin{document}

  \setlength{\sectionskip}{\baselineskip}
  \setlength{\itemsep}{0pt}
  \setlength{\parskip}{0.7\baselineskip}


% Center the name over the entire width of resume:
 \moveleft.5\hoffset\centerline{\large\bf Nathan W. Brei}
 \moveleft.5\hoffset\centerline{n\_brei@alum.mit.edu}
 \moveleft.5\hoffset\centerline{(617) 460-9743}
 % Draw a horizontal line the whole width of resume:
 \moveleft\hoffset\vbox{\hrule width\resumewidth height 1pt}\smallskip
% address begins here
% Again, the address lines must be centered over entire width of resume:
\begin{resume}


\section{EDUCATION}% 
\textbf{Massachusetts Institute of Technology}, Cambridge, MA \\
Bachelor's of Science in Aerospace Engineering, June 2011\\
\underline{Awards:} Lufthansa Prize for Excellence in German Studies, 3rd Place.\\
\textbf{Coursework:}\\
\hspace*{.3in}\underline{Applied math:} ODEs, Linear Algebra, Computational Science and Engineering\\
\hspace*{.3in}\underline{Engineering:} Numerical methods, Structural Mechanics, Aerodynamics\\
\hspace*{.3in}\underline{Architecture:} Design studios 1 \& 2; History and Theory of Architecture


\textbf{Independent and Internet-based study}\\
\underline{Coursera:} Artificial Intelligence, Machine Learning, Finite Automata, Computer Vision\\
\underline{Udacity:} Programming a Robotic Car



\textbf{International School of D\"{u}sseldorf}, %
                D\"{u}sseldorf, Germany \\
                International Baccalaureate, June 2007. National Merit Finalist.


\section{EXPERIENCE}

 {\bf Makani Power}, \emph{Design Engineering Intern \hfill November 2012 - Present} \\
 Designing and implementing part of the ground station for an airborne wind turbine. The system tethers the wing to the ground, acting as an anchor, a winch, a perch, and a power station. Focused on the mechanical design of structural elements and chain drives. Emphasis on rapid prototyping; CAD design work (SolidWorks, NX) and FEM analysis (CosmosWorks, Nastran).

 {\bf Daedalus Innovation}, \emph{Backend Web Developer \hfill September - November 2012} \\
 Contributed to BookAndTalk.com, a soon-to-be-launched startup bringing together book groups and famous writers. Coded core functionality in Python/Django and integrated with external packages including Zinnia CMS and OpenTok. Managed the Postgresql database and the Heroku deployment.

 {\bf Fairfield Foundation}, 
 \emph{Archaeology Intern \hfill July - September 2012}\\
Archaeological fieldwork at four colonial sites in Tidewater Virginia. Surveyed, dug, cleaned, interpreted, and mapped test units and shovel tests. Sorted, cleaned, and catalogued artifacts. Did archival research on a historic estate. Helped renovate a 1920's gas station.

 {\bf 16.622: Senior capstone project} \hfill \emph{Fall 2010 - Spring 2011} \\
Designed and implemented a yearlong experiment, with a partner, under faculty guidance. Applied machine learning techniques to ultrawideband radio waveforms to improve localization performance. Tested ability to identify the material of barriers blocking the line of sight. 

 {\bf MIT Technology Laboratory for Advanced Materials and Structures } \\
 \emph{Undergraduate Research Opportunity Program \hfill Summer 2010}\\
Designed, machined, and tested a device to hold individual carbon fibers under tension inside a CVD furnace. Grew aligned carbon nanotube forests. Prepared carbon fibers with various coatings and tested their tensile strength. Prepared and cured graphite-epoxy composites.

 {\bf MIT Daylighting Lab}, \emph{Undergraduate Research Opportunity Program \hfill Summer 2008}\\
Coded a daylighting simulation plugin, as part of a team, for SketchUp using Ruby and Java. Unified code and results from multiple master's theses. The plugin, `LightSolve', calculates natural lighting characteristics and interactively assists with design optimization.


\section{SKILLS} 
\begin{description} \itemsep -2pt
    \item[Programming:] Proficient: Java, C, Matlab, Python. Familiar: Clojure, Scheme, Ruby, Arduino.
    \item[Linguistic:] Fluent in German.
    \item[Design:] SolidWorks, Rhino, NX, and \& SketchUp CAD. Manual drafting, sketching, \LaTeX.
    \item[Machine Shop:] Waterjet, MIG welder, 3D printer, lathe, mill, etc.
\end{description}


\section{INTERESTS}
Applied math, operations research, computational design, optimization, computer vision, thin-tile vaulted structures. Drawing, Shotokan karate, caving, swing dancing.
\end{resume}
\end{document}
